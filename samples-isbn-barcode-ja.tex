\documentclass[a5j,tombo,10pt,titlepage,pdfusetitle]{ltjsbook}

% LINK
\usepackage{url}
\usepackage{hyperref}
\hypersetup{pdfborder={0 0 0.5}}

% 色の使用
\usepackage{xcolor}
\definecolor{mylinkcolor}{RGB}{3, 112, 145} %{65, 145, 3} % 色定義
\definecolor{linkcol}{RGB}{2, 106, 77} %{65, 145, 3} % 色定義
\hypersetup{
    colorlinks=true,
    citecolor=blue,
    linkcolor=linkcol,
    urlcolor=mylinkcolor % 定義された色
}
% HTML(#800000)色定義
\def\colH#1{\color[HTML]{#1}}

% 画像挿入
\usepackage{float}
\usepackage{graphicx}
\usepackage{svg} 

\usepackage{pst-barcode} 

% FONT
\usepackage{fontspec}
% FONT-WEIGHT 調整の指定
% \usepackage[haranoaji,deluxe,nfssonly,jis2004]{luatexja-preset}
\usepackage[haranoaji,deluxe]{luatexja-preset}

% MACRO
\def\ocrb#1{\fontspec{ocrb7} {#1}}
\def\ocrbsmall#1{\fontspec{ocrb7} \fontsize{7}{7}{#1}\selectfont }

\newfontfamily\fOCRBRegular{OCRB-Regular}
\usepackage{luatexja-otf} % UNICODE


\usepackage{framed}

% FONT-SIZE
\def\fs#1#2{\fontsize{#1}{#2}\selectfont }
\def\bf#1{\textbf{#1}}

\title{{\Huge 書籍JANコード}\\ \lbrack サンプル集 \rbrack\vspace{60mm}}

\author{\href{https://github.com/ru-museum?tab=repositories}{ru\_museum}: GitHub)}
\date{\today}

\begin{document}
\thispagestyle{empty}

\maketitle

% \setcounter{tocdepth}{2}
% \clearpage
% \thispagestyle{empty}

% \tableofcontents

\newpage
\thispagestyle{empty}

\section*{EXAMPLES}   
\begin{itemize}
  \item 以下掲載のサンプルは、実際の書籍の表示を再現しています。
  \item 円マーク文字には、書籍により{\colH{800000}\bfseries¥}と{\colH{800000}\jfontspec{OCRB-Regular} \UTF{00A5}}とが用いられています。\\
  「{\colH{800000}\jfontspec{OCRB-Regular} \UTF{00A5}}」の記述方法は\href{https://github.com/ru-museum/isbn-barcode-ja-latex/isbn-barcode-ja-latex.pdf}{isbn-barcode-ja-latex.pdf}を参照して下さい。
\end{itemize}

\section*{サンプル書籍}   
    
\begin{enumerate}\fs{9}{14}
  \item \textbf{再発見 日本の哲学 丸山眞男─理念への信}:遠山敦 著、講談社、2010
  \item \textbf{戦争と平和 4}:トルストイ 著 / 藤沼 貴 訳、岩波文庫、2006 
  \item \textbf{夢見る権利 ロシア・アヴァンギャルド再考}:桑野隆、東京大学出版会、1996 
  \item \textbf{岩波講座 言語の科学 9 言語情報処理}:長尾 真 他著、岩波書店、1998 
  \item \textbf{やさしいC}:高橋麻奈 著、ソフトバンク クリエイティブ、2012 
\end{enumerate}


\newpage
\thispagestyle{empty}

\begin{picture}(0,0)(0,-10)
\put(0,120){EXAMPLE: 1}   
\end{picture}  

\begin{picture}(0,0)(0,20)

\put(-33,76){\psbarcode{9784062787611}{height=0.45 width=1.28}{ean13}}  
\put(-34,66){{\fontspec{ocrb7}9784062787611}}  
\put(-33,13){\psbarcode{1920310014001}{height=0.45 width=1.28}{ean13}}  
\put(-34,3){{\fontspec{ocrb7}1920310014001}}  

\put(92,100){\fontspec{Inter-Medium}ISBN978-4-06-278761-1}  
\put(92,86){\fontspec{Inter-Medium}C0310 \gtfamily{\bfseries¥}1400E (0)}  
\put(92,36){\fs{11}{13}{\fontspec{NotoSerifJP-Bold}定価:本体 {\fontspec{GoudyOldStyleSSiSmallCaps}1400}円(税別)}}  
\put(92,16){\fs{11}{12}{\fontspec{NotoSerifJP-Bold}講談社}}  
\end{picture}  

\vspace{100mm}

\begin{framed}
{\fs{14}{10} \noindent\bf{再発見 日本の哲学 丸山眞男─理念への信}}\\

{\fs{12}{10}
\noindent 遠山 敦 著}\vspace{0mm}\\

\noindent 講談社 2010年
\end{framed}



\newpage
\thispagestyle{empty}

\begin{picture}(0,0)(0,-10)
\put(0,120){EXAMPLE: 2}   
\end{picture}  

\begin{picture}(0,0)(0,20)

\put(-33,76){\psbarcode{9784003261842}{height=0.45 width=1.28}{ean13}}  
\put(-34,66){{\fontspec{ocrb7}9784003261842}}  
\put(-33,13){\psbarcode{1920197009404}{height=0.45 width=1.28}{ean13}}  
\put(-34,3){{\fontspec{ocrb7}1920197009404}}  

\put(92,100){\fontspec{ocrb7}ISBN4-00-326184-4}  

% \put(92,78){\fontspec{ocrb7}C0197 \gtfamily{\bfseries¥}940E}  
\put(92,78){\fontspec{ocrb7}C0197 {\jfontspec{OCRB-Regular} \UTF{00A5}}\hspace{-0.6mm}940E}  
\put(92,46){\fs{12}{12}{\gtfamily{\mdseries 定価(本体 {\fontspec{Inter-Medium}940}円+税)}}}
\end{picture}\\  

\vspace{80mm}

\begin{framed}
{\fs{14}{10} \noindent\bf{戦争と平和 4}}\\

{\fs{12}{10}
\noindent トルストイ 作\\ 
 藤沼 貴 訳}\vspace{0mm}\\

\noindent 赤 618-4\\
岩波文庫
\end{framed}

\newpage
\thispagestyle{empty}

\begin{picture}(0,0)(0,-10)
\put(0,120){EXAMPLE: 3}   
\end{picture}  

\begin{picture}(0,0)(0,20)

\put(238,76){\psbarcode{9784130130196}{height=0.45 width=1.28}{ean13}}  
\put(237,66){{\fontspec{ocrb7}9784130130196}}  
\put(238,13){\psbarcode{1911010029877}{height=0.45 width=1.28}{ean13}}  
\put(237,3){{\fontspec{ocrb7}1911010029877}}  

\put(87,100){\fontspec{ocrb7}ISBN4-13-013019-6}  
\put(87,78){\fontspec{ocrb7}C1010 P2987E}  
\put(93,30){\fs{12}{12}\gtfamily{\mdseries 定価{\fontspec{Inter-Medium}2987}円}}  
\put(87.8,16){\fs{12}{12}\gtfamily{\mdseries ({\fontspec{Inter-Medium}本体2900円})}}

%\gtfamily{\mdseries 定価(本体{\fontspec{Inter-Medium}3400}


  
\end{picture}  

\vspace{100mm}

\begin{framed}
{\fs{14}{10} \noindent\bf{夢見る権利 ロシア・アヴァンギャルド再考}}\\
{\fs{12}{10}\noindent 桑野 隆 著}\vspace{0mm}\\
\noindent 東京大学出版会 1996\\
\end{framed}

\newpage
\thispagestyle{empty}

\begin{picture}(0,0)(0,-10)
\put(0,120){EXAMPLE: 4}   
\end{picture}  

\begin{picture}(0,0)(0,20)

\put(238,76){\psbarcode{9784000108591}{height=0.45 width=1.28}{ean13}}  
\put(237,66){{\fontspec{ocrb7}9784000108591}}  
\put(238,13){\psbarcode{1923380034009}{height=0.45 width=1.28}{ean13}}  
\put(237,3){{\fontspec{ocrb7}1923380034009}}  

\put(87,100){\fontspec{ocrb7}ISBN4-00-010859-X}  
\put(87,78){\fontspec{ocrb7}C3380 {\jfontspec{OCRB-Regular} \UTF{00A5}}\hspace{-0.8mm}3400E}  
\put(87,30){\fs{12}{12}{\gtfamily{\mdseries 岩波書店}}}  
\put(87,16){\fs{8.2}{12}{\gtfamily{\mdseries 定価(本体{\fontspec{Inter-Medium}3400}円+税)}}}  
\end{picture}  

\vspace{100mm}

\begin{framed}
{\fs{14}{10} \noindent\bf{岩波講座 言語の科学 9 言語情報処理}}\\
{\fs{12}{10}\noindent 長尾 真 他著}\vspace{0mm}\\
\noindent 岩波書店 1998\\
\end{framed}

\newpage
\thispagestyle{empty}

\begin{picture}(0,0)(0,-10)
\put(0,120){EXAMPLE: 5}   
\end{picture}  

\begin{picture}(0,0)(0,20)

\put(238,76){\psbarcode{9784797370980}{height=0.45 width=1.28}{ean13}}  
\put(237,66){{\fontspec{ocrb7}9784797370980}}  
\put(238,13){\psbarcode{1920055025003}{height=0.45 width=1.28}{ean13}}  
\put(237,3){{\fontspec{ocrb7}1920055025003}}  

\put(80,100){\fs{10}{9} \fontspec{Inter-Medium}ISBN4-7973-7098-0}  
\put(80,84){\fs{10}{9} \fontspec{Inter-Medium}C1010 \gtfamily{\bfseries¥}2500E}  
\put(82,32){\fs{12}{12}{\gtfamily{\mdseries 定価 \fbox{本体{\fontspec{Inter-Medium}2,500}円}+税}}}  
\end{picture}  

\vspace{100mm}

\begin{framed} 
{\fs{14}{10} \noindent\bf{やさしいC}}\\
{\fs{12}{10}\noindent 高橋 麻奈 著}\vspace{0mm}\\
\noindent ソフトバンク クリエイティブ 2012\\
\end{framed}

\end{document}

